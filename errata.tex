\documentclass{iopart}
\usepackage[T1]{fontenc}
\usepackage[latin9]{inputenc}
\usepackage{amstext}

\makeatletter

%%%%%%%%%%%%%%%%%%%%%%%%%%%%%% LyX specific LaTeX commands.
%% Because html converters don't know tabularnewline
\providecommand{\tabularnewline}{\\}

%%%%%%%%%%%%%%%%%%%%%%%%%%%%%% Textclass specific LaTeX commands.
\usepackage{iopams}
\usepackage{setstack}
\usepackage{enumitem}		% customizable list environments
\newlength{\lyxlabelwidth}      % auxiliary length 

%%%%%%%%%%%%%%%%%%%%%%%%%%%%%% User specified LaTeX commands.
\usepackage{amsfonts}
\usepackage{amssymb}
\usepackage{amsbsy}
\usepackage[usenames,dvipsnames]{xcolor}
\usepackage{braket}%Added
\usepackage{dsfont}%Added

%%%  Definitions for theorem-like environments. Don't change them, and don't make new ones.
\newtheorem{theorem}{}[section]
\newtheorem{corollary}[theorem]{}\newtheorem{lemma}[theorem]{}\newtheorem{proposition}[theorem]{}\newtheorem{definition}[theorem]{}\let\olddefinition\definition
\renewcommand{\definition}{\olddefinition\normalfont}
%
\newenvironment{remark}{\addtocounter{theorem}{1}\par\noindent\textit{Remark \thetheorem}\normalfont}{\par}
% \newenvironment{example}{\addtocounter{theorem}{1}\par\noindent\textit{Example \thetheorem}\normalfont}{\par}
\newenvironment{notation}{\par\noindent\textit{Notation }\normalfont}{\\[1ex]}
%
% \newenvironment{proof}{\par\noindent\textit{Proof }}{\hfill\ensuremath{\Box}\par}
%
%%%  A few standard macros that we can all use
%
\newcommand{\R}{\mathbb{R}}
\newcommand{\C}{\mathbb{C}}
\newcommand{\Z}{\mathbb{Z}}
\newcommand{\Q}{\mathbb{Q}}
\newcommand{\ComplexI}{\text{i}}
%
%
\newcommand{\JGF}{\mathbb{F}}    %Added
%
%
%
\newcommand{\ff}[1]{\mathbb{F}_{#1}}
\newcommand{\ip}[2]{\langle #1 ~|~ #2 \rangle}
\def\fh{\mathfrak{h}}
\newcommand{\yutsung}[1]{\fbox{\begin{minipage}{0.9\textwidth}\color{purple}{Yu-Tsung says: #1}\end{minipage}}}
\addtocounter{equation}{39}
\addtocounter{section}{5}
\addtocounter{subsection}{3}

\makeatother

\begin{document}

\title{Erratum: Geometry of discrete quantum computing}


\author{Andrew J. Hanson}


\address{School of Informatics and Computing, Indiana University, Bloomington,
IN 47405, U.S.A}


\author{Gerardo Ortiz}


\address{Department of Physics, Indiana University, Bloomington, IN 47405,
U.S.A}


\author{Amr Sabry}


\address{School of Informatics and Computing, Indiana University, Bloomington,
IN 47405, U.S.A}


\author{Yu-Tsung Tai}


\address{Department of Mathematics, Indiana University, Bloomington, IN 47405,
U.S.A}


\address{School of Informatics and Computing, Indiana University, Bloomington,
IN 47405, U.S.A}


\noindent{\it Keywords\/}: {\today}

\maketitle

\noindent We correct Sec.~5.4 of our paper~\cite{geometry}. There are several related mistakes that all stem from an incorrect use of equation~(27) in a discrete setting with modular arithmetic. Equation~(27) defines purity from which the definitions of unentangled, partially entangled, and maximally entangled states were derived in the field of complex numbers. In finite fields, the definitions are more subtle because of modular arithmetic as explained below:

\paragraph*{Purity, Product States, and Unentangled States.} The paper correctly identifies unentangled states with product states. In the continuous case, a state has maximal purity iff it is a product state. In the discrete case, product states have maximal purity, but the converse does not necessarily hold. 
\paragraph*{Purity and Maximally Entangled States.} The paper defines maximally entangled states as those whose purity is 0. This definition needs to be modified in the discrete case to avoid identifying as maximally entangled states whose purity is a multiple of the field characteristic.

\bigskip The results in other sections do not depend on equation~(27): the corrected version of Sec.~5.4 below has no influence on the rest of the article. 

\subsection{Maximal entanglement}

In the case of finite fields, we need to modify the definition of purity because modular arithmetic confuses some of the identifications. For instance, although product states and maximal purity can be identified when using complex arithmetic, there exist, in the discrete case, non-product states that also have maximal purity. 

The first step in adapting equation~(27) to the discrete case is to eliminate the $1/n$ factor to avoid dividing by zero whenever $n$ is a multiple of the characteristic of the field:
\begin{equation}
\overline{P_{\fh}}=\sum\limits _{j=1}^{n}\overline{P_{\fh}^{j}}, \qquad\mbox{with~}
\overline{P_{\fh}^j} = \sum\limits _{\mu=x,y,z}\langle\sigma_{\mu}^{j}\rangle^{2}\ ,
\end{equation}
where $\overline{P_{\fh}}\in\ff{p}$. 

This change simply re-scales the definition of purity: 


For example, consider the normalized product state $\ket{\Psi_{0}}$ and the normalized entangled 
state $\ket{\Psi_{1}}$ over $\ff{3^{2}}$: 
\begin{eqnarray}
\ket{\Psi_{0}} & = & \ket{0}\otimes\frac{\ket{0}-\ket{1}}{1-\ComplexI}\otimes\frac{\ket{0}-\ket{1}}{1-\ComplexI}\ ,\\
\ket{\Psi_{1}} & = & \ket{011}+\ket{100}+\ket{101}+\ket{110}\ .
\end{eqnarray}
In both cases $\overline{P_{\fh}^j} = 1$ for all $j$: the states are not distinguished by purity. 

$\ket{\Psi_{0}}$ also serves an example that a product state has
$\overline{P_{\fh}}=0$ because of modular arithmetic. In the continuous
case, $\overline{P_{\fh}}=0$ is equivalent to 
\begin{equation}
\forall j,\forall\mu,\braket{\sigma_{\mu}^{j}}=0\ .\label{eq:john}
\end{equation}
In discrete case, it turns out that equation (\ref{eq:john}) is the
one defines maximally entangled states. From this definition, the
Bell states are maximally entangled, while the normalized state $\ket{\Psi_{2}}=\ket{01}+\ket{10}+\left(-1-\ComplexI\right)\ket{11}$
with $\overline{P_{\fh}^{j}}=0$ for all $j$ over $\ff{3^{2}}$ is
not maximally entangled.

Computing some examples for various $n$ and small values of $p$,
one can verify explicitly that unit-norm product states for $n=2$,
$p=\{3,7,11,19,\ldots\}$ occur with frequency 
\begin{equation}
(p+1)p^{2}(p-1)^{2}=\{144,14\,112,145\,200,2339\,280,\ldots\}\ ,
\end{equation}
and for general $n$, $(p+1)p^{n}(p-1)^{n}$.

The irreducible state counts are reduced by $(p+1)$, giving 
\begin{equation}
p^{2}(p-1)^{2}=\{36,1764,12\,100,116\,964,\ldots\}\ ,
\end{equation}
and in general for $n$-qubits, there are $p^{n}\left(p-1\right)^{n}$ instances
of product pure states.

Repeating the computation, we find maximally entanglement states with 
frequencies for two qubits of 
\begin{equation}
p\left(p^{2}-1\right)\left(p+1\right)=\left\{ 96,2688,15\,840,136\,800,\ldots\right\} \ .
\end{equation}
The irreducible state counts for maximal entanglement are reduced
by $\left(p+1\right)$, giving for $n=2$
\begin{equation}
p\left(p^{2}-1\right)=\left\{ 24,336,1320,6840,\ldots\right\} \ .
\end{equation}
For three-qubits, there are $p^{3}\left(p^{4}-1\right)\left(p+1\right)$
instances of pure maximally entangled states. For four-qubits and
$p=3$, there are $2195\,538\,048$ instances of pure maximally entangled
states, while the general formula for four-qubit states remains unclear.

\yutsung{
\begin{eqnarray*}
2195\,538\,048 & = & 4\cdot548\,884\,512\ ,\\
548\,884\,512 & = & 2^{5}\cdot3^{7}\cdot11\cdot23\cdot31\ .
\end{eqnarray*}
The irreducible state counts for maximal entanglement for $n=4$ and
$p=7$ is about $8.22\times10^{14}$.}


\ack{}{We would like to thank John Gardiner for pointing out the need for this correction. This research was supported in part by Lilly Endowment, Inc., through its support for the Indiana University Pervasive Technology Institute, and in part by the Indiana METACyt Initiative. The Indiana METACyt Initiative at IU is also supported in part by Lilly Endowment, Inc.}


\section*{References}{}

\bibliographystyle{plain}
\bibliography{gardinerbib}

\end{document}
