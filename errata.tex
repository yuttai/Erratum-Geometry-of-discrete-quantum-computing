\documentclass{iopart}
\usepackage[T1]{fontenc}
\usepackage[latin9]{inputenc}
\usepackage{amstext}

\makeatletter

%%%%%%%%%%%%%%%%%%%%%%%%%%%%%% LyX specific LaTeX commands.
%% Because html converters don't know tabularnewline
\providecommand{\tabularnewline}{\\}

%%%%%%%%%%%%%%%%%%%%%%%%%%%%%% Textclass specific LaTeX commands.
\usepackage{iopams}
\usepackage{setstack}
\usepackage{enumitem}		% customizable list environments
\newlength{\lyxlabelwidth}      % auxiliary length 

%%%%%%%%%%%%%%%%%%%%%%%%%%%%%% User specified LaTeX commands.
\usepackage{amsfonts}
\usepackage{amssymb}
\usepackage{amsbsy}
\usepackage[usenames,dvipsnames]{xcolor}
\usepackage{braket}%Added
\usepackage{dsfont}%Added

%%%  Definitions for theorem-like environments. Don't change them, and don't make new ones.
\newtheorem{theorem}{}[section]
\newtheorem{corollary}[theorem]{}\newtheorem{lemma}[theorem]{}\newtheorem{proposition}[theorem]{}\newtheorem{definition}[theorem]{}\let\olddefinition\definition
\renewcommand{\definition}{\olddefinition\normalfont}
%
\newenvironment{remark}{\addtocounter{theorem}{1}\par\noindent\textit{Remark \thetheorem}\normalfont}{\par}
% \newenvironment{example}{\addtocounter{theorem}{1}\par\noindent\textit{Example \thetheorem}\normalfont}{\par}
\newenvironment{notation}{\par\noindent\textit{Notation }\normalfont}{\\[1ex]}
%
% \newenvironment{proof}{\par\noindent\textit{Proof }}{\hfill\ensuremath{\Box}\par}
%
%%%  A few standard macros that we can all use
%
\newcommand{\R}{\mathbb{R}}
\newcommand{\C}{\mathbb{C}}
\newcommand{\Z}{\mathbb{Z}}
\newcommand{\Q}{\mathbb{Q}}
\newcommand{\ComplexI}{\text{i}}
%
%
\newcommand{\JGF}{\mathbb{F}}    %Added
%
%
%
\newcommand{\ff}[1]{\mathbb{F}_{#1}}
\newcommand{\ip}[2]{\langle #1 ~|~ #2 \rangle}
\def\fh{\mathfrak{h}}
\newcommand{\yutsung}[1]{\fbox{\begin{minipage}{0.9\textwidth}\color{purple}{Yu-Tsung says: #1}\end{minipage}}}
\addtocounter{equation}{39}
\addtocounter{section}{5}
\addtocounter{subsection}{3}

\makeatother

\begin{document}

\title{Erratum: Geometry of discrete quantum computing}


\author{Andrew J. Hanson}


\address{School of Informatics and Computing, Indiana University, Bloomington,
IN 47405, U.S.A}


\author{Gerardo Ortiz}


\address{Department of Physics, Indiana University, Bloomington, IN 47405,
U.S.A}


\author{Amr Sabry}


\address{School of Informatics and Computing, Indiana University, Bloomington,
IN 47405, U.S.A}


\author{Yu-Tsung Tai}


\address{Department of Mathematics, Indiana University, Bloomington, IN 47405,
U.S.A}


\address{School of Informatics and Computing, Indiana University, Bloomington,
IN 47405, U.S.A}


\noindent{\it Keywords\/}: {\today}

\maketitle

\noindent Sec.~5.4 of our paper~\cite{geometry} requires a clarification and a correction. 

\paragraph*{Clarification: Unentangled vs. product states.} In conventional quantum mechanics, using the real and complex numbers, a state is unentangled when it can be expressed as a product state or when equation~(27) reports its purity to be 1. When using finite fields, it is possible for equation~(27) to produce a purity of 1 for some entangled states. For example, consider the 3-qubit state \ldots



Thus, in finite fields, the simplest way to calculate the number of unentangled states is to ignore equation~(27) and to count the number of product states. This is exactly how the counting in Sec.~5.4 was produced but the paper did not clarify that counting using equation~(27) would be incorrect.

\paragraph*{Correction: Maximally entangled states.} In conventional quantum mechanics, a state is maximally entangled when equation~(27) reports its purity to be 0. In the discrete case, states whose purity is a multiple of the field characteristic are incorrectly labeled as maximally entangled. For example, \ldots

To correctly account for maximally entangled states, we modify equation~(27) as follows:

\ldots

\ack{}{We would like to thank John Gardiner for pointing out the need for this correction.}

\section*{References}{}

\bibliographystyle{plain}
\bibliography{gardinerbib}

\end{document}
