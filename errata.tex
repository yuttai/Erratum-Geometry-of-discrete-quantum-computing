\documentclass[12pt]{iopart}
\usepackage[T1]{fontenc}
\usepackage[latin9]{inputenc}
\usepackage{amstext}

\makeatletter

%%%%%%%%%%%%%%%%%%%%%%%%%%%%%% LyX specific LaTeX commands.
%% Because html converters don't know tabularnewline
\providecommand{\tabularnewline}{\\}

%%%%%%%%%%%%%%%%%%%%%%%%%%%%%% Textclass specific LaTeX commands.
\usepackage{iopams}
\usepackage{setstack}
\usepackage{enumitem}		% customizable list environments
\newlength{\lyxlabelwidth}      % auxiliary length 

%%%%%%%%%%%%%%%%%%%%%%%%%%%%%% User specified LaTeX commands.
\usepackage{amsfonts}
\usepackage{amssymb}
\usepackage{amsbsy}
\usepackage[usenames,dvipsnames]{xcolor}
\usepackage{braket}%Added
\usepackage{dsfont}%Added

%%%  Definitions for theorem-like environments. Don't change them, and don't make new ones.
\newtheorem{theorem}{}[section]
\newtheorem{corollary}[theorem]{}\newtheorem{lemma}[theorem]{}\newtheorem{proposition}[theorem]{}\newtheorem{definition}[theorem]{}\let\olddefinition\definition
\renewcommand{\definition}{\olddefinition\normalfont}
%
\newenvironment{remark}{\addtocounter{theorem}{1}\par\noindent\textit{Remark \thetheorem}\normalfont}{\par}
% \newenvironment{example}{\addtocounter{theorem}{1}\par\noindent\textit{Example \thetheorem}\normalfont}{\par}
\newenvironment{notation}{\par\noindent\textit{Notation }\normalfont}{\\[1ex]}
%
% \newenvironment{proof}{\par\noindent\textit{Proof }}{\hfill\ensuremath{\Box}\par}
%
%%%  A few standard macros that we can all use
%
\newcommand{\R}{\mathbb{R}}
\newcommand{\C}{\mathbb{C}}
\newcommand{\Z}{\mathbb{Z}}
\newcommand{\Q}{\mathbb{Q}}
%
%
\newcommand{\JGF}{\mathbb{F}}    %Added
%
%
%
\newcommand{\ff}[1]{\mathbb{F}_{#1}}
\newcommand{\ip}[2]{\langle #1 ~|~ #2 \rangle}
\def\fh{\mathfrak{h}}
\newcommand{\yutsung}[1]{\fbox{\begin{minipage}{0.9\textwidth}\color{purple}{Yu-Tsung says: #1}\end{minipage}}}
\addtocounter{equation}{39}
\addtocounter{section}{5}
\addtocounter{subsection}{3}

\makeatother

\begin{document}

\title{Erratum: Geometry of discrete quantum computing}

\author{Andrew J. Hanson$^{1}$, Gerardo Ortiz$^{2}$, Amr Sabry$^{1}$,
Yu-Tsung Tai$^{3,1}$}


\address{$^{1}$ School of Informatics and Computing, Indiana University,
Bloomington, IN 47405, U.S.A}


\address{$^{2}$ Department of Physics, Indiana University, Bloomington, IN
47405, U.S.A}


\address{$^{3}$ Department of Mathematics, Indiana University, Bloomington,
IN 47405, U.S.A}


\ead{ortizg@indiana.edu}

\submitto{\jpa}

\maketitle

\noindent Section~5.4 of our paper~\cite{geometry} requires a clarification and a correction.

\paragraph*{Clarification: Unentangled vs.\ product states.}
In conventional quantum mechanics, using the field of complex numbers, a state is unentangled when it can be expressed as a product state or, equivalently, when equation~(27) reports its purity to be 1~\cite{PhysRevLett.92.107902, PhysRevA.68.032308}. When using finite Galois fields $\ff{p^2}$, for particular choices of $p$, it is possible for equation~(27) to produce a purity of 1 for some entangled states. For example, the normalized entangled state $\ket{\Psi} = \ket{011}+\left(2+\rmi\right)\ket{100}+\ket{101}+\ket{110}$ has $P_{\fh}=1$ for $p=7$. In addition, the process of determining whether a \emph{given} state $\ket{\Phi}$ is a product state may depend on $p$. 

Thus, in finite fields, the simplest way to calculate the number of
unentangled states is to disregard equation~(27) and count the number
of product states. This is exactly how the counting in section~5.4
was done, but the paper did not point out that counting relying only on equation~(27) might not lead to the same result.

\paragraph*{Correction: Maximally entangled states.}
We present below a new version of section~5.4 that correctly counts maximally entangled states. The rest of the article is independent of this revision.

\subsection{Maximal entanglement}

Equation~(27) for $P_{\fh}$ includes a normalization factor $\frac{1}{n}$. In the discrete case, this normalization factor is undefined when $p ~|~ n$. Equation~(27) also includes a summation of $n$ terms. In the discrete case, certainly when $p ~|~ n$ but also in other cases, this summation may vanish in the field even if the individual summands are non-zero. These anomalies are irrelevant for the classification of unentangled states as this computation is performed by directly checking the possibility of direct decomposition into product states, disregarding equation~(27). 

For maximally entangled states, the purity calculation in conventional quantum mechanics using equation~(27) produces 0. Given the above observations, in a discrete field, equation~(27) may be undefined or may report a purity of 0 even for partially entangled states. For example, the normalized 5-qubit state $\ket{\Psi}=\left(1-\rmi\right)\left(\ket{00}+\ket{11}\right)\otimes\ket{000}$ has $P_{\fh}=0$ for $p=3$, and is not maximally entangled because only the first two qubits are entangled. In the discrete case, we therefore check for maximally entangled states using the following equations,
\begin{equation}
\forall j,\forall\mu,\braket{\sigma_{\mu}^{j}}^2 =0\textrm{ ,}\label{eq:john}
\end{equation}
which avoids the normalization factor and simply checks that each summand is 0. 

We now implement these procedures to enumerate the maximally entangled states for the specific cases for $n = 2 ,3 $ and compare these to the counts for product states. We can verify explicitly that the numbers of unit-norm product states for $n=2$,
$p=\{3,7,11,19,\ldots\}$ are  
\[
(p+1)p^{2}(p-1)^{2}=\{144,14\,112,145\,200,2339\,280,\ldots\}\ ,
\]
and for general $n$, \[(p+1)p^{n}(p-1)^{n}\ .\]

The irreducible state counts are reduced by $(p+1)$, giving 
\[
p^{2}(p-1)^{2}=\{36,1764,12\,100,116\,964,\ldots\}\ ,
\]
and in general for $n$-qubits, there are $p^{n}\left(p-1\right)^{n}$
instances of pure product states.

Performing the computation using equation (\ref{eq:john}), we find the numbers of maximally entangled states for two qubits to be 
\[
p\left(p^2-1\right)\left(p+1\right)=\left\{ 96,2688,15\,840,136\,800,\ldots\right\} \ .
\]
The irreducible state counts for maximal entanglement are reduced
by $\left(p+1\right)$, giving, for $n=2$, 
\[
p\left(p^{2}-1\right)=\left\{ 24,336,1320,6840,\ldots\right\} \ .
\]
For three qubits, there are $p^{3}\left(p^{4}-1\right)\left(p+1\right)$ (total) and $p^{3}\left(p^{4}-1\right)$ (irreducible) instances of pure maximally entangled states, while the general formula for 4-qubit states remains unclear.

Therefore, the ratio of maximally entangled to product states is
\[
\frac{\textbf{Max entangled}}{\textbf{Product}}=\frac{p+1}{p\left(p-1\right)} ,\, \frac{\left(p^2+1\right)\left(p+1\right)}{\left(p-1\right)^2} \ ,
\]
for $n=2$ and $3$, respectively.

\ack{}{We would like to thank John Gardiner for pointing out the
need for this correction.}


\section*{References}

{}

\bibliographystyle{iopart-num}
\bibliography{gardinerbib}

\end{document}
